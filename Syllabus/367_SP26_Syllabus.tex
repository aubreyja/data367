\documentclass{article}
\usepackage{amsmath}
\usepackage[margin=1in]{geometry}
\usepackage[pdftex]{graphicx}
\usepackage{hyperref}
\usepackage[parfill]{parskip}
\usepackage{xcolor}
%\setlength{\oddsidemargin}{0in}
%\setlength{\topmargin}{0.5in}
%\setlength{\textwidth}{6in}
%\setlength{\textheight}{9in}

\newcommand{\room}{ENGR 304}
\newcommand{\meetingtime}{3-4:15pm}
\newcommand{\final}{Thursday December 18 10:30am--12:30pm}

\title{DATA 367 -- Statistical Methods in Sports Analytics\\ \small \room\  MW \meetingtime\ -- Spring 2026\vspace{-1.5cm}}
\author{}
\date{}
\begin{document}
\maketitle

{\bf About the Course:} This course will introduce statistical methods and training in statistical consulting aimed to
analyze sports by using observational data on players and teams. With an emphasis on statistical
inference and modeling, the students will learn how to analyze a sports related problem, utilize
statistical tools to find a solution and interpret those results to sports professionals. The course
will also offer the opportunity to focus on a semester long sports analytics project.

	{\bf Course Prerequisites:} MATH 129 or MATH 263

It is also recommended that students have experience in a programming language. We will primarily use Python in this class, but no experience with
Python is necessary to succeed in the class.


	{\bf Instructor and Contact Information}

Primary Instructor: Jason Aubrey (jaubrey@arizona.edu)

Office hours: Mondays 1-2pm in Math 220, Tuesdays and Wednesdays 1-2pm in Math 219

Graduate Teaching Assistant: Jeffrey Mei (jmei@arizona.edu)

Office Hours: Mondays and Wednesdays in PAS 526

Our D2L page will serve as the course home page - check it often. Course materials will be published on D2L and important announcements will be made on D2L.

	{\bf Course Format and Teaching Methods}

This class is scheduled to be taught in the in-person modality.

	{\bf Course Communications}

Announcements and important course information may be sent out via official University email or through D2L. It is the student's responsibility to check for messages and announcements regularly. Email should and will be used for notification purposes - however, it is a poor tool for discussion. Mathematical questions should be asked, and will be discussed, in our class meetings - either during class (or after class if time permitting), during office hours or by appointment.

	{\bf Required Texts or Readings}

There is no required text for the course. Supplementary reading material will be provided to students via D2L.

	{\bf Required or Special Materials}

You are required to bring a laptop to every class meeting.

This course will use the programming language Python, Jupyter notebooks, Quarto, git and github, and Positron. You can obtain this software as follows:\\
\begin{itemize}
	\item Python - \url{http://python.org}
	\item Jupyter - \url{http://jupyter.org}
	\item Quarto - \url{http://quarto.org}
	\item Positron - \url{https://positron.posit.co/}
	\item Git - \url{http://git-scm.com}
	\item Github - \url{http://github.com}
\end{itemize}

It is your responsibility to have this software installed on the laptop you bring to class. The websites above have documentation and downloads for the most
common operating systems, such as MacOSX, Windows, and many Linux distributions. If you run an unusual or niche operating system, you may have to install this
software from source.

The software above is what we will use in lecture. Assignments will be turned in as Quarto Markdown (QMD) files often by making a commit to a
git repository. If you are an advanced student, in some cases you may use R, and you may use an IDE other than Positron, but we will not be
teaching R in this class, and we may not be able to help you with other IDEs.

Equipment and software requirements: For this class you MUST bring a laptop to class with a reliable internet signal that can:
\begin{itemize}
	\item Access D2L
	\item Run the software above
	\item Access Gradescope
	\item Scan and upload written work to Gradescope
	\item View pdf documents
\end{itemize}

{\bf Class Meetings}

Meeting Times: This class will meet on Mondays and Wednesdays from 3-4:15 pm in ENGR 304. Our meetings will give us the opportunity to develop our understanding of the ideas and methods of Sports Analytics. Most days we will utilize group work. Expect to turn in a sample of group work to be graded almost every class meeting.

	{\bf Class attendance:}
\begin{itemize}
	\item If you feel sick, or if you need to isolate or quarantine based on University protocols, stay home. Except for seeking medical care, avoid contact with others and do not travel.
	\item Notify your instructor if you will be missing a course meeting or an assignment deadline.
	\item Non-attendance for any reason does not guarantee an automatic extension of due date or rescheduling of examinations. Please communicate and coordinate any request directly with your instructor.
	\item If you must miss the equivalent of more than one week of class, you should contact the Dean of Students Office DOS-deanofstudents@email.arizona.edu to share documentation about the challenges you are facing.
\end{itemize}
Class Recordings: For lecture recordings, which are used at the discretion of the instructor, students must access content in D2L only. Students may not modify content or re-use content for any purpose other than personal educational reasons. All recordings are subject to government and university regulations. Therefore, students accessing unauthorized recordings or using them in a manner inconsistent with UArizona values and educational policies are subject to suspension or civil action.

	{\bf Absence and Class Participation Policy}

Participating in the course and attending lectures and other course events are vital to the learning process. As such, attendance is required at all lectures and discussion section meetings. If you anticipate being absent, are unexpectedly absent, or are unable to participate in class activities, please contact me as soon as possible. Students who miss the first two class meetings, and do not contact me within 24 hours of the second class meeting, may be administratively dropped.

	{\bf Accessibility and Accommodations}

At the University of Arizona, we strive to make learning experiences as accessible as possible. If you anticipate or experience barriers based on disability or pregnancy, please contact the Disability Resource Center (520-621-3268, \url{https://drc.arizona.edu/}) to establish reasonable accommodations.

	{\bf Classroom Behavior Policy}

To foster a positive learning environment, students and instructors have a shared responsibility. We want a safe, welcoming, and inclusive environment where all of us feel comfortable with each other and where we can challenge ourselves to succeed. To that end, our focus is on the tasks at hand and not on extraneous activities (e.g., texting, chatting, reading a newspaper, making phone calls, web surfing, etc.).

	{\bf Your responsibilities as a class member}
\begin{itemize}
	\item Be fully engaged in the mathematics, with your peers while in the classroom. This means put aside non-math conversations, texting, social media, and anything else that may make this time less mathematically productive for you and your peers.
	\item Be ready and willing to participate in many different forms of interactive activities, including small-group discussion, explaining ideas and R-code to others, working out code individually and in a group, and adding/modifying other's solution code.
	\item Listen to your peers' arguments and the instructor's lead discussion(s) respectfully, politely and engagedly - be willing and ready to contribute whenever appropriate.
	\item Come to class mentally prepared, so that you (and your peers) may benefit from being in an interactive class.
	\item Be on time and ready to start right when class is scheduled to start, and remain until the class is dismissed.
\end{itemize}

{\bf Course Objectives and Expected Learning Outcomes}

This course will have students utilize statistical tools to solve sports analytics problems, including but not limited to, factors influencing game outcomes
and individual performance metrics. This course will begin with an examination of the history of sports analytics and continue on to discuss how to numerically
and visually analyze sports related data. The course will also present methods of evaluating team and player performance data using a variety of
techniques, including data visualization, regression and hypothesis testing. An emphasis will be placed on learning how to describe outcomes from
analysis in a non-technical manner.

Along with the learning outcomes from this course, all students will participate in a project to assist in the data collection and analysis from one of the participating University of Arizona athletic programs. Students will use analysis techniques learned in class to provide expected outcomes based on coach input, while also improving on current techniques in data collection and analysis to provide new insights for the coaching staff to utilize.

By the end of this course, students will be able to:
\begin{itemize}
	\item Utilize statistical tools to solve sports analytics problems, including but not limited to, factors influencing game outcomes and individual performance metrics.
	\item Numerically and visually analyze sports related data.
	\item Utilize methods of evaluating team and player performance data using a variety of techniques, including data visualization, regression and hypothesis testing.
	\item Describe outcomes from analysis of sports-related data in a non-technical manner.
\end{itemize}

{\bf Assignments and Examinations: Schedule/Due Dates}

In Class Group Activities - 50 points - almost every day

Research Paper - 75 points - week 9

Applied Statistical Analysis in Sports Paper - 75 points - week 13

5 Python Assignments - 100 points - weeks 3, 4, 5, 6, 7

Team Sport Presentation 60 points - week 14

Final Results Paper - 50 points - week 16

Final Results Presentation 80 points - week 16 or 17

	%Reflection on Project - 10 points - week 17

	{\bf In Class Group Activities}

Graded group activities will be assigned during class meetings. Many will involve Python programming - others will involve project updates. According to the calendar, we expect to have 27 graded class activities, each worth 10 points. In general, no make-up activities will be offered. When computing your final class activities grade (out of 50 points) we will take the total number of class activities points earned and divide by 230 and multiply by 50 (to a maximum of 50 ).

	{\bf Python Assignments}

Graded Python assignments will be assigned in the first half of the course. These assignments will build on class instruction and activities and will be group
assignments. It is expected that groups of $3-5$ students will work together, and deductions may be applied for failing to contribute
appropriately to the group work, or not submitting the assignment properly or on time. There will be 5 Python assignments in the semester, each worth 20 points.

{\bf Research Paper}

The purpose of this paper is to analyze a positional sports paper or article that includes advanced analytics (such as WAR, PER, QBR, Real plus/minus, etc. ). It should be a $3-5$ page paper that includes:
\begin{itemize}
	\item A summary of the author's thesis and argument.
	\item A thorough examination of the most important analytical part of the article.
	      \begin{itemize}
		      \item A thorough examination (and explanation) of this analysis/metric. This should be very detailed - walk your reader through how to calculate it. If you can get hold of the data used and reconstruct the analysis or the metric, this would be best.
		      \item What are the positive features of this analysis/metric? How is it an improvement over previous methods/metrics?
		      \item What are the negative features of this analysis/metric? What improvements need to be made?
		      \item How does this analysis/metric supports the author's argument? Are there any issues with using this analysis/metric to support their argument?
	      \end{itemize}
	\item A brief description of any other analytics in the paper. Include a brief comment on how the analysis/metric supports the author's argument and if there are any issues with using this analysis/metric.
\end{itemize}

{\bf Applied Statistical Analysis in Sports Paper}

The purpose of this paper is to use the techniques learned in Math 367 (or elsewhere) to explore an idea of your own in sports analytics. It should be a $2-4$ page paper (longer if needed).
\begin{itemize}
	\item Sport: Pick whichever sport you would like. Give enough information about the sport so that the reader can understand the terms you use your thesis question.
	\item Thesis: Pick whatever question you would like to explore. Be sure to discuss the potential impact an answer would have on the sport.
	\item Data: It is up to you to find data to work with. Obviously, this may limit the questions you may be able to explore. If you want to explore the impact of going for it on fourth down in college football, and are planning to write a simulation that samples from fourth down plays, then trying to collect all fourth down plays in NCAA Division I football history may be too ambitious. Start with a simulation on a much more limited set, do the appropriate analysis, reach the appropriate conclusion (which would be more limited), and then discuss how one would improve the simulation by increasing the data set in the areas of improvement part of the paper.
	\item Analysis: Be sure to include some advanced analytic:
	      \begin{itemize}
		      \item it could be a metric (such as WAR, PER, etc...)
		      \item it could be a method (such as regression, a simulation, clustering, or hypothesis testing, etc...)
		      \item Whatever metric or method you choose (and it could be more than one), it should be appropriate for the question.
		      \item Go through your metric or method thoroughly. The reader should be able to reproduce your results.
		      \item For example, if you use WAR, it is clear how one calculates WAR (and the reader could calculate WAR for other players by following your work).
		      \item Use comments in your code to describe what the Python commands you use are doing.
	      \end{itemize}
	\item Conclusion: The conclusion you reach should be appropriate to the methods and results you get. (If you do a hypothesis test and do not meet the $5 \%$ threshold, then concluding ``we do not have enough evidence to conclude that...'' is the appropriate conclusion.) Do not overreach. Do not make statements here that are not supported by your work.
	\item Areas of Improvement: Discuss ideas for areas of improvement. This could include expanding your data sets, or starting to collect data that nobody has collected yet, or improvements to a simulation to add more realism, or ideas for new metrics that haven't been thought of yet, or...
	\item Format: Your paper should be submitted as an Quarto markdown file (QMD). You should also submit any data files necessary to run your Python code. You will lose points on your paper if the Python code does not successfully execute for the grader.
\end{itemize}

{\bf Final Project}

A final project, consisting of four parts, will be completed in the last half of the class. The four parts are: the Team Sport Presentation, the Final Results Presentation, the Final Results Paper, and the Reflection on Project. The final projects are group projects, and you will be given significant class time to work on a project - however, you will also need to allocate time outside of class to complete the project. Failure to be present or participate during days allocated to project work, may impact your grade on the Final Results Presentation.

The Team Sport Presentation will be given on the 14th week, which your team will:
\begin{itemize}
	\item Introduce your sport, and the overall project that you are undertaking.
	\item Identify the semester goal the team has for this project.
	\item Build benchmarks for the entire project which specify each task that must be completed
	\item Discuss the deadline for each task
	\item Identify work as assigned to each team member
\end{itemize}

The Final Results Paper will be due May 6th (the last day of class)
\begin{itemize}
	\item This paper should include all of the relevant work on your project. It should include all the analysis that supports the conclusions in the project presentations. If you think of the presentation as the ``highlights'' of the project, think of this paper as where the audience goes to find all the details.
	\item Your paper should be submitted as an Quarto markdown file (QMD). You should also submit any data files necessary to run your Python code. You will lose points on your paper if the Python code does not successfully execute for the grader.
\end{itemize}

Project presentations will be completed the last two weeks of class (including the time scheduled for our final exam, May 12th from 3:30-5:30 pm) and will:
\begin{itemize}
	\item Discuss the motivation of the project and how it can assist the team for which it was designed
	\item Detail any significant results
	\item If you have code, demonstrate how it works
	\item If you performed analysis, show all results and what they mean (This section should take up the most time and you should translate results for coaches to understand)
	\item Wrap up your presentation by detailing how current results can be utilized
	\item Discuss recommendations for future work in this area
\end{itemize}

The Reflection on Project will be due May 6th. It is a one to two single-spaced typed page document where you are asked to reflect and describe your work and discoveries on the sports analytics project this semester. Be sure to include a description of your contributions and also your thoughts on how or whether this activity helped your professional development or influenced your professionalism in team work and collaboration, communication, and problem solving.

If you have more than two absences during days devoted to project work, your final project grade will be reduced by $10 \%$, and more than four absences on those days your final project grade will be reduced by $20 \%$, and more than six absences on those days, your final project grade will be reduced by $30 \%$.

Please note the following:
\begin{itemize}
	\item University rules relating to final examinations may be found at: \url{https://www.registrar.arizona.edu/courses/final-examination-regulations-and-information}
	\item The University final exam schedule may be found at: \url{http://www.registrar.arizona.edu/students/courses/final-exams}
\end{itemize}

{\bf Grading Scale and Policies}

\begin{table}
	\begin{tabular}{|l|l|}
		\hline \multicolumn{2}{|l|}{Categories}                              \\
		\hline In Class Group Activites                     & 50 pts (10\%)  \\
		\hline Research Paper                               & 75 pts (15 \%) \\
		\hline Applied Statistical Analysis in Sports Paper & 75 pts (15\%)  \\
		\hline $5$ Python Assignments                       & 100 pts (20\%) \\
		\hline Reflection                                   & 10 pts (2\%)   \\
		\hline Team Sport Presentation                      & 60 pts (12\%)  \\
		\hline Final Results Paper                          & 50 pts (10\%)  \\
		\hline Final Results Presentation                   & 80 pts (16 \%) \\
		\hline
	\end{tabular}
	%\captionsetup{labelformat=empty}
	\caption{Total possible points}
\end{table}

{\bf You will earn a grade of:}

A if you earn at least 450 points ( $90 \%$ )

B if you earn at least 400 points ( $80 \%$ )

C if you earn at least 350 points (70\%)

D if you earn at least 300 points (60\%)

Requests for incomplete (I) or withdrawal (W) must be made in accordance with University policies, which are available at
\begin{itemize}
	\item \url{http://catalog.arizona.edu/policy/grades-and-grading-system\#incomplete}
	\item \url{http://catalog.arizona.edu/policy/grades-and-grading-system\#Withdrawal respectively}.
\end{itemize}

You may drop the class without a W through January 27 using UAccess. The class will appear on your UAccess record, but will not appear on your transcript. You may withdraw with a W through March 31 using UAccess. The University allows withdrawals through April 14, but only with the Dean's approval. Late withdraws are dealt with on a case by case basis, and requests for late withdraw without a valid reason may or may not be granted.

Administrative Drops: Administrative drop is an instructor's option, not an obligation. Instructors are not required to drop students who fail to attend class. Since students may add courses beyond the official start date, instructors should be attentive to student enrollment dates when assessing adequate participation for the purposes of administrative drop. Students may be administratively dropped if they miss the first two class meetings.

Dispute of Grade Policy: In general, any questions regarding the grading of any assignment, quiz, or exam need to be cleared up within one week after the graded item has been returned.

	{\bf Confidentiality of Student Records}
\url{http://www.registrar.arizona.edu/personal-information/family-educational-rights-and-privacy-act-1974-ferpa?topic=ferpa}

{\bf Subject to Change Statement}

Information contained in the course syllabus, other than the grade and absence policy, may be subject to change with advance notice, as deemed appropriate by the instructor.

	{\bf Scheduled Topics/Activities}

\begin{tabular}{|l|l|l|l|}
	\hline Week & Dates      & Topics Covered                               & Assessments                                       \\
	\hline 1    & 1/14       & Course Introduction                          &                                                   \\
	\hline 2    & 1/21       & Linear Regression                            &                                                   \\
	\hline 3    & 1/26, 1/28 & Linear Regression                            & Python Assignment \#1                             \\
	\hline 4    & 2/2, 2/4   & Logistic Regression                          & Python Assignment \#2                             \\
	\hline 5    & 2/9, 2/11  & Getting Data                                 & Python Assignment \#3                             \\
	\hline 6    & 2/16, 2/18 & Manipulating and Processing Data             & Python Assignment \#4                             \\
	\hline 7    & 2/23, 2/25 & Simulations                                  & Python Assignment \#5                             \\
	\hline 8    & 3/2, 3/4   & Bootstrapping Introduction to Data Analytics & Research Paper                                    \\
	\hline      & 3/9, 3/11  & Spring Break!                                                                                    \\
	\hline 9    & 3/16, 3/18 & Hypothesis Testing                           &                                                   \\
	\hline 10   & 3/23, 3/25 & Group Meetings                               &                                                   \\
	\hline 11   & 3/30, 4/01 & Group Meetings                               &                                                   \\
	\hline 12   & 4/6, 4/8   & Group Presentations - Introduce your Sport   & Team Sport Presentation                           \\
	\hline 13   & 4/13, 4/15 & Group Presentations                          & Applied Statistical Analysis in Sports Paper      \\
	\hline 14   & 4/20, 4/22 & Group Meetings                               &                                                   \\
	\hline 15   & 4/27, 5/29 & Group Meetings                               &                                                   \\
	\hline 16   & 5/04, 5/06 & Final Presentations                          & Final Results Presentation \& Reflection          \\
	\hline 17   & 5/12       & Final Presentations from 3:30-5:30pm         & Final Results Presentation \& Final Results Paper \\
	\hline
\end{tabular}

{\bf Additional Resources for Students}

UA Academic policies and procedures are available at \url{http://catalog.arizona.edu/policies}
Student Assistance and Advocacy information is available at \url{http://deanofstudents.arizona.edu/student-assistance/students/student-assistance}

%Academic advising: If you have questions about your academic progress this semester, please reach out to your academic advisor (https://advising.arizona.edu/advisors/major). Contact the Advising Resource Center (https://advising.arizona.edu/) for all general advising questions and referral assistance. Call 520-626-8667 or email_to advising@.arizona.edu
%Life challenges: If you are experiencing unexpected barriers to your success in your courses, please note the Dean of Students Office is a central support resource for all students and may be helpful. The Dean of Students Office can be reached at 520-621-2057 or DOS-deanofstudents@email.arizona.edu.

Physical and mental-health challenges: If you are facing physical or mental health challenges this semester, please note that Campus Health provides quality medical
and mental health care. For medical appointments, call (520) 621-9202. For After Hours care, call (520) 570-7898. For the Counseling \& Psych Services (CAPS) 24/7 hotline, call (520) 621-3334.

{\bf University-wide Policies link}

The Links to the following UA policies are provided here, \url{https://academicaffairs.arizona.edu/syllabus-policies}:
\begin{itemize}
	\item Absence and Class Participation
	\item Policy Regarding Absences for Any Sincerely Held Religious Belief, Observance or Practice
	\item Threatening Behavior Policy
	\item Accessibility and Accommodations
	\item Code of Academic Integrity
	\item Nondiscrimination and Anti-Harassment Policy
\end{itemize}

\end{document}
